% Options for packages loaded elsewhere
% Options for packages loaded elsewhere
\PassOptionsToPackage{unicode}{hyperref}
\PassOptionsToPackage{hyphens}{url}
\PassOptionsToPackage{dvipsnames,svgnames,x11names}{xcolor}
%
\documentclass[
  letterpaper,
  DIV=11,
  numbers=noendperiod]{scrartcl}
\usepackage{xcolor}
\usepackage{amsmath,amssymb}
\setcounter{secnumdepth}{-\maxdimen} % remove section numbering
\usepackage{iftex}
\ifPDFTeX
  \usepackage[T1]{fontenc}
  \usepackage[utf8]{inputenc}
  \usepackage{textcomp} % provide euro and other symbols
\else % if luatex or xetex
  \usepackage{unicode-math} % this also loads fontspec
  \defaultfontfeatures{Scale=MatchLowercase}
  \defaultfontfeatures[\rmfamily]{Ligatures=TeX,Scale=1}
\fi
\usepackage{lmodern}
\ifPDFTeX\else
  % xetex/luatex font selection
\fi
% Use upquote if available, for straight quotes in verbatim environments
\IfFileExists{upquote.sty}{\usepackage{upquote}}{}
\IfFileExists{microtype.sty}{% use microtype if available
  \usepackage[]{microtype}
  \UseMicrotypeSet[protrusion]{basicmath} % disable protrusion for tt fonts
}{}
\makeatletter
\@ifundefined{KOMAClassName}{% if non-KOMA class
  \IfFileExists{parskip.sty}{%
    \usepackage{parskip}
  }{% else
    \setlength{\parindent}{0pt}
    \setlength{\parskip}{6pt plus 2pt minus 1pt}}
}{% if KOMA class
  \KOMAoptions{parskip=half}}
\makeatother
% Make \paragraph and \subparagraph free-standing
\makeatletter
\ifx\paragraph\undefined\else
  \let\oldparagraph\paragraph
  \renewcommand{\paragraph}{
    \@ifstar
      \xxxParagraphStar
      \xxxParagraphNoStar
  }
  \newcommand{\xxxParagraphStar}[1]{\oldparagraph*{#1}\mbox{}}
  \newcommand{\xxxParagraphNoStar}[1]{\oldparagraph{#1}\mbox{}}
\fi
\ifx\subparagraph\undefined\else
  \let\oldsubparagraph\subparagraph
  \renewcommand{\subparagraph}{
    \@ifstar
      \xxxSubParagraphStar
      \xxxSubParagraphNoStar
  }
  \newcommand{\xxxSubParagraphStar}[1]{\oldsubparagraph*{#1}\mbox{}}
  \newcommand{\xxxSubParagraphNoStar}[1]{\oldsubparagraph{#1}\mbox{}}
\fi
\makeatother

\usepackage{color}
\usepackage{fancyvrb}
\newcommand{\VerbBar}{|}
\newcommand{\VERB}{\Verb[commandchars=\\\{\}]}
\DefineVerbatimEnvironment{Highlighting}{Verbatim}{commandchars=\\\{\}}
% Add ',fontsize=\small' for more characters per line
\usepackage{framed}
\definecolor{shadecolor}{RGB}{241,243,245}
\newenvironment{Shaded}{\begin{snugshade}}{\end{snugshade}}
\newcommand{\AlertTok}[1]{\textcolor[rgb]{0.68,0.00,0.00}{#1}}
\newcommand{\AnnotationTok}[1]{\textcolor[rgb]{0.37,0.37,0.37}{#1}}
\newcommand{\AttributeTok}[1]{\textcolor[rgb]{0.40,0.45,0.13}{#1}}
\newcommand{\BaseNTok}[1]{\textcolor[rgb]{0.68,0.00,0.00}{#1}}
\newcommand{\BuiltInTok}[1]{\textcolor[rgb]{0.00,0.23,0.31}{#1}}
\newcommand{\CharTok}[1]{\textcolor[rgb]{0.13,0.47,0.30}{#1}}
\newcommand{\CommentTok}[1]{\textcolor[rgb]{0.37,0.37,0.37}{#1}}
\newcommand{\CommentVarTok}[1]{\textcolor[rgb]{0.37,0.37,0.37}{\textit{#1}}}
\newcommand{\ConstantTok}[1]{\textcolor[rgb]{0.56,0.35,0.01}{#1}}
\newcommand{\ControlFlowTok}[1]{\textcolor[rgb]{0.00,0.23,0.31}{\textbf{#1}}}
\newcommand{\DataTypeTok}[1]{\textcolor[rgb]{0.68,0.00,0.00}{#1}}
\newcommand{\DecValTok}[1]{\textcolor[rgb]{0.68,0.00,0.00}{#1}}
\newcommand{\DocumentationTok}[1]{\textcolor[rgb]{0.37,0.37,0.37}{\textit{#1}}}
\newcommand{\ErrorTok}[1]{\textcolor[rgb]{0.68,0.00,0.00}{#1}}
\newcommand{\ExtensionTok}[1]{\textcolor[rgb]{0.00,0.23,0.31}{#1}}
\newcommand{\FloatTok}[1]{\textcolor[rgb]{0.68,0.00,0.00}{#1}}
\newcommand{\FunctionTok}[1]{\textcolor[rgb]{0.28,0.35,0.67}{#1}}
\newcommand{\ImportTok}[1]{\textcolor[rgb]{0.00,0.46,0.62}{#1}}
\newcommand{\InformationTok}[1]{\textcolor[rgb]{0.37,0.37,0.37}{#1}}
\newcommand{\KeywordTok}[1]{\textcolor[rgb]{0.00,0.23,0.31}{\textbf{#1}}}
\newcommand{\NormalTok}[1]{\textcolor[rgb]{0.00,0.23,0.31}{#1}}
\newcommand{\OperatorTok}[1]{\textcolor[rgb]{0.37,0.37,0.37}{#1}}
\newcommand{\OtherTok}[1]{\textcolor[rgb]{0.00,0.23,0.31}{#1}}
\newcommand{\PreprocessorTok}[1]{\textcolor[rgb]{0.68,0.00,0.00}{#1}}
\newcommand{\RegionMarkerTok}[1]{\textcolor[rgb]{0.00,0.23,0.31}{#1}}
\newcommand{\SpecialCharTok}[1]{\textcolor[rgb]{0.37,0.37,0.37}{#1}}
\newcommand{\SpecialStringTok}[1]{\textcolor[rgb]{0.13,0.47,0.30}{#1}}
\newcommand{\StringTok}[1]{\textcolor[rgb]{0.13,0.47,0.30}{#1}}
\newcommand{\VariableTok}[1]{\textcolor[rgb]{0.07,0.07,0.07}{#1}}
\newcommand{\VerbatimStringTok}[1]{\textcolor[rgb]{0.13,0.47,0.30}{#1}}
\newcommand{\WarningTok}[1]{\textcolor[rgb]{0.37,0.37,0.37}{\textit{#1}}}

\usepackage{longtable,booktabs,array}
\usepackage{calc} % for calculating minipage widths
% Correct order of tables after \paragraph or \subparagraph
\usepackage{etoolbox}
\makeatletter
\patchcmd\longtable{\par}{\if@noskipsec\mbox{}\fi\par}{}{}
\makeatother
% Allow footnotes in longtable head/foot
\IfFileExists{footnotehyper.sty}{\usepackage{footnotehyper}}{\usepackage{footnote}}
\makesavenoteenv{longtable}
\usepackage{graphicx}
\makeatletter
\newsavebox\pandoc@box
\newcommand*\pandocbounded[1]{% scales image to fit in text height/width
  \sbox\pandoc@box{#1}%
  \Gscale@div\@tempa{\textheight}{\dimexpr\ht\pandoc@box+\dp\pandoc@box\relax}%
  \Gscale@div\@tempb{\linewidth}{\wd\pandoc@box}%
  \ifdim\@tempb\p@<\@tempa\p@\let\@tempa\@tempb\fi% select the smaller of both
  \ifdim\@tempa\p@<\p@\scalebox{\@tempa}{\usebox\pandoc@box}%
  \else\usebox{\pandoc@box}%
  \fi%
}
% Set default figure placement to htbp
\def\fps@figure{htbp}
\makeatother


% definitions for citeproc citations
\NewDocumentCommand\citeproctext{}{}
\NewDocumentCommand\citeproc{mm}{%
  \begingroup\def\citeproctext{#2}\cite{#1}\endgroup}
\makeatletter
 % allow citations to break across lines
 \let\@cite@ofmt\@firstofone
 % avoid brackets around text for \cite:
 \def\@biblabel#1{}
 \def\@cite#1#2{{#1\if@tempswa , #2\fi}}
\makeatother
\newlength{\cslhangindent}
\setlength{\cslhangindent}{1.5em}
\newlength{\csllabelwidth}
\setlength{\csllabelwidth}{3em}
\newenvironment{CSLReferences}[2] % #1 hanging-indent, #2 entry-spacing
 {\begin{list}{}{%
  \setlength{\itemindent}{0pt}
  \setlength{\leftmargin}{0pt}
  \setlength{\parsep}{0pt}
  % turn on hanging indent if param 1 is 1
  \ifodd #1
   \setlength{\leftmargin}{\cslhangindent}
   \setlength{\itemindent}{-1\cslhangindent}
  \fi
  % set entry spacing
  \setlength{\itemsep}{#2\baselineskip}}}
 {\end{list}}
\usepackage{calc}
\newcommand{\CSLBlock}[1]{\hfill\break\parbox[t]{\linewidth}{\strut\ignorespaces#1\strut}}
\newcommand{\CSLLeftMargin}[1]{\parbox[t]{\csllabelwidth}{\strut#1\strut}}
\newcommand{\CSLRightInline}[1]{\parbox[t]{\linewidth - \csllabelwidth}{\strut#1\strut}}
\newcommand{\CSLIndent}[1]{\hspace{\cslhangindent}#1}



\setlength{\emergencystretch}{3em} % prevent overfull lines

\providecommand{\tightlist}{%
  \setlength{\itemsep}{0pt}\setlength{\parskip}{0pt}}



 


\KOMAoption{captions}{tableheading}
\makeatletter
\@ifpackageloaded{caption}{}{\usepackage{caption}}
\AtBeginDocument{%
\ifdefined\contentsname
  \renewcommand*\contentsname{Table of contents}
\else
  \newcommand\contentsname{Table of contents}
\fi
\ifdefined\listfigurename
  \renewcommand*\listfigurename{List of Figures}
\else
  \newcommand\listfigurename{List of Figures}
\fi
\ifdefined\listtablename
  \renewcommand*\listtablename{List of Tables}
\else
  \newcommand\listtablename{List of Tables}
\fi
\ifdefined\figurename
  \renewcommand*\figurename{Figure}
\else
  \newcommand\figurename{Figure}
\fi
\ifdefined\tablename
  \renewcommand*\tablename{Table}
\else
  \newcommand\tablename{Table}
\fi
}
\@ifpackageloaded{float}{}{\usepackage{float}}
\floatstyle{ruled}
\@ifundefined{c@chapter}{\newfloat{codelisting}{h}{lop}}{\newfloat{codelisting}{h}{lop}[chapter]}
\floatname{codelisting}{Listing}
\newcommand*\listoflistings{\listof{codelisting}{List of Listings}}
\makeatother
\makeatletter
\makeatother
\makeatletter
\@ifpackageloaded{caption}{}{\usepackage{caption}}
\@ifpackageloaded{subcaption}{}{\usepackage{subcaption}}
\makeatother
\usepackage{bookmark}
\IfFileExists{xurl.sty}{\usepackage{xurl}}{} % add URL line breaks if available
\urlstyle{same}
\hypersetup{
  pdftitle={EuropeListening},
  colorlinks=true,
  linkcolor={blue},
  filecolor={Maroon},
  citecolor={Blue},
  urlcolor={Blue},
  pdfcreator={LaTeX via pandoc}}


\title{EuropeListening}
\author{}
\date{}
\begin{document}
\maketitle


\section{Introduction}\label{introduction}

The European Union (EU) has long envisioned itself as a global champion
of democracy, human rights, and the rule of law, projecting an identity
of `normative power' that underpins its foreign policy and international
engagement (\citeproc{ref-manners2002normative}{Manners, 2002}). This
self-image is reflected in key policy documents such as the European
Democracy Action Plan in 2020 and the Defence of Democracy package in
2023, both framing the EU as a `force for good' and a committed promoter
of democracy. Central to this vision is the EU's aspiration to be viewed
as a legitimate and credible actor, with its influence largely derived
from its soft power and the appeal of its values
(\citeproc{ref-nye2004soft}{Nye, 2004}). However, historical examples
highlight a disconnect between the EU's self-perception and its external
image. In Tunisia, for instance, post-2011 EU democracy promotion was
criticized for neglecting local visions and priorities
(\citeproc{ref-pace2014eu}{Pace, 2014}), while in the Eastern
Partnership countries the EU's agenda has sometimes alienated local
elites as neo-imperialist (\citeproc{ref-delcour2013meandering}{Delcour,
2013}).

Scholarly research supports that some political elites outside Europe do
not necessarily share the EU's positive self-image
(\citeproc{ref-chaban_east_2007}{Natalia Chaban \& Kauffmann, 2007};
\citeproc{ref-elgstrom_outsiders_2007}{Elgstrom, 2007};
\citeproc{ref-scheipers_normative_2007}{Scheipers \& Sicurelli, 2007};
\citeproc{ref-schlipphak_action_2013}{Schlipphak, 2013, p. 590}). Such
misperception can undermine EU initiatives, leading to resistance,
strained relations, and limited policy uptake, ultimately affecting the
EU's legitimacy and its ability to promote and defend democracy
(\citeproc{ref-smith2011enlargement}{Smith, 2011};
\citeproc{ref-smith2014european}{2014, p. 209}). It is reasonable to
assume that EU personnel involved in external relations are aware of how
their policies are perceived in target countries and might adjust their
practices to maximize influence. However, existing research has focused
almost exclusively on how the EU is perceived by political elites and
citizens outside Europe, leaving a significant gap in understanding how
EU representatives themselves perceive the EU's external image and how
these perceptions shape their (preferred) actions. This knowledge gap
persists despite long-standing calls by International Relations (IR)
scholars (e.g., \citeproc{ref-holsti1970national}{K. Holsti, 1970};
\citeproc{ref-jervis1976}{Jervis, 1976};
\citeproc{ref-sprout1957environmental}{Sprout \& Sprout, 1957}) for more
systematic study of both the psychological and operational environments
of policy actions. As (\citeproc{ref-wish1980foreign}{Wish, 1980, p.
532}) noted, there are still ``few systematic empirical studies
centering on decision makers' perception of their own nations''.
Understanding how meta-perceptions shape EU democracy promotion
practices can potentially help to improve the alignment between EU
objectives and the realities on the ground, thereby enhancing democracy
support.

In this paper, we examine how EU officials and representatives from key
institutions (`EU representatives') perceive the EU's external image
(that is, how they believe the EU and its foreign policy are viewed by
their counterparts in partner countries) and how these meta-perceptions
shape the diplomatic practices they consider appropriate for promoting
democracy abroad. Rather than focusing on the substance of the
democratic model the EU seeks to export or on the actual implementation
of EU policies, we investigate the preferences of EU representatives
regarding specific diplomatic instruments. In doing so, we shift the
analytical focus from high-level policy formulation, typically driven by
political leaders, to the ground-level perspectives of the EU's
diplomatic staff, public servants and members of parliament involved in
advancing and supporting democracy promotion (cf.
\citeproc{ref-cooper2013oxford}{Cooper, Heine, \& Thakur, 2013, p. 2}).
Our interest lies in how their preferences are shaped by their
interpretations of how the EU and its democracy promotion is perceived
externally. In particular, we ask whether shared perceptions of the EU's
external image give rise to ``communities of practice''
(\citeproc{ref-bicchi2018}{Bicchi \& Bremberg, 2018, pp. 9--10}), in
which groups of EU representatives converge around similar
understandings of what constitutes appropriate and legitimate democracy
promotion. The central research question guiding our analysis is thus:
\emph{How do EU representatives' understandings of the EU's external
image influence democracy promotion practices they prefer to be used
towards partner countries?}

The literature on EU external perceptions paints a complex and often
critical picture of how the EU is viewed from the outside, highlighting
several key challenges for legitimate and effective foreign policy.
First, EU external action receives little media coverage and remains
largely unknown among the general public outside the EU {[}N. Chaban \&
Holland (\citeproc{ref-chaban2008}{2008})\}. `Europe' is more often
associated with individual member states or the geographic continent
than with the EU as a political actor. Second, the EU is typically
perceived as an economic giant prioritizing security and economic
interests rather than as a principled foreign political leader
(\citeproc{ref-elgstrom_outsiders_2007}{Elgstrom, 2007}). Third, its
external actions are often viewed as inconsistent and inefficient,
undermining its credibility and impact (\citeproc{ref-lucarelli2010}{S.
Lucarelli \& Fioramonti, 2010}). In this study, we ask whether EU
representatives are aware of these challenges, and how such perceived
limitations shape their understanding of what diplomatic practices are
appropriate in their interactions with external counterparts.

Furthermore, the existing literature on EU external perceptions has long
been criticized for being predominantly descriptive, with increasing
calls for more explanatory approaches (e.g.,
\citeproc{ref-lucarelli_seen_2014}{Sonia Lucarelli, 2014}). While recent
work has begun to address this gap, it largely continues to focus on how
perceptions are formed rather than on their consequences for the EU's
foreign policy legitimacy and effectiveness. A notable exception is
emerging research in EU climate and environmental diplomacy, which shows
that external perceptions can shape EU foreign policy outcomes, either
by generating expectations that constrain effectiveness when unmet
(\citeproc{ref-torney2014external}{Torney, 2014}), or by creating
uncertainty and reducing coherence when internal and external views
diverge (\citeproc{ref-delreux2021ego}{Delreux \& Pipart, 2021}). At the
same time, scholarship associated with the sociological `practice turn'
in International Relations (IR) (e.g., \citeproc{ref-adler2011}{Adler \&
Pouliot, 2011}) emphasizes how diplomacy is constituted through
habitual, socially embedded practices, but often overlooks how
perceptions or reputational considerations feed into these practices. By
examining how EU representatives' perceptions of the EU's external image
shape their preferred diplomatic practices, our study brings these two
strands of research into productive dialogue.

In the next section, we formulate hypotheses on the link between
meta-perceptions and preferred practices drawing on the
psychologically-informed literature on (meta-)perceptions and diplomatic
practices. We test these hypotheses using original data from one
specific case: EU officials and representatives from key institutions in
charge of relations with the Eastern neighborhood countries. We use a
modified version of Q-methodology to identify how these EU
representatives think the EU is perceived by their counterparts and
explore how their views shape their practices based on a quantitative
survey. Q-methodology is particularly suited to produce a comprehensive
view of an individual's viewpoint and to uncover subjective
understandings (\citeproc{ref-brewer2000}{Brewer, Selden, \& Facer,
2000}; \citeproc{ref-brown2008}{Steven R. Brown, Durning, \& Selden,
2008}; \citeproc{ref-stephenson1953}{Stephenson, 1953}), but has been
underutilized in the field of international relations and social
sciences more broadly. In the specific context of our study, the method
prevents EU officials and representatives from key institutions from
repeating official policy lines, which is a limitation in previous
research about belief systems and external images. Overall, this paper
adds the perceptual and practical perspectives of ``those agents
involved in the quotidian unfolding''
(\citeproc{ref-pouliot2010}{Pouliot, 2010, p. 1}) of democracy
promotion, in an attempt to identify the micro-foundations of effective,
legitimate European democracy support that can inform concrete
recommendations for the development of democracy promotion advocacy
coalitions between the EU and the Eastern neighborhood region.

\section{External perceptions and diplomatic
practices}\label{external-perceptions-and-diplomatic-practices}

The study of international relations and foreign policy has
traditionally focused on general patterns of conflict and cooperation,
often emphasizing system-level variables such as power dynamics,
security concerns, and economic interdependence. However, beneath these
observable patterns lies the critical role of individual decision-makers
who often act ``based less on their objective circumstances than their
perceptions of those circumstances''
(\citeproc{ref-renshon2008theory}{Renshon \& Renshon, 2008, p. 511}).
This insight underpins cognitive process models, which stress that
foreign policy choices are shaped more by how actors interpret and
process information than by any `objective' reality.

Early work in this tradition emphasizes the cognitive foundations of
decision-making in foreign policy
(\citeproc{ref-hermann1967attempt}{Hermann \& Hermann, 1967};
\citeproc{ref-holsti1962belief}{O. Holsti, 1962}). Building on these
classics, Shapiro \& Bonham (\citeproc{ref-shapiro1973cognitive}{1973})
argues that it is individual perceptions, rather than the actual
geopolitical environment, that drive behavior. Along similar lines, K.
Holsti (\citeproc{ref-holsti1970national}{1970}) introduces the concept
of national role conceptions, which refers to decision-makers'
understanding of their nation's role in the international system.
Encompassing beliefs about a nation's responsibilities, commitments, and
obligations, K. Holsti (\citeproc{ref-holsti1970national}{1970, p. 243})
asserts that these conceptions play a fundamental role in guiding
foreign policy decisions. They not only shape how states see themselves
but also how they engage with others. Overall, early IR scholars argue
that policy- and decision-makers are influenced not by the `objective'
facts of the situation, whatever they may be, but by their own `image'
of it (\citeproc{ref-boulding1959national}{Boulding, 1959, p. 120}).

Closely linked to insights about role conceptions is the concept of
reputation. In international affairs, marked by uncertainty and
incomplete information, reputation serves as a critical guide for
strategic behavior. As Dafoe, Renshon, \& Huth
(\citeproc{ref-dafoe2014reputation}{2014, p. 365}) observes, states and
organizations often make strategic decisions based on how others are
perceived to view them, particularly because direct knowledge of
intentions is limited. Reputation extends beyond an `objective' record
of past behavior; it reflects beliefs and judgments within the
international community about an actor's identity, trustworthiness, and
predictability (\citeproc{ref-dafoe2014reputation}{Dafoe et al., 2014};
\citeproc{ref-jervis1989logic}{Jervis, 1989}). This subjective nature of
reputation makes it a powerful tool in international diplomacy,
influencing negotiations, cooperation, and conflict resolution. A strong
reputation, as Keohane (\citeproc{ref-keohane2005after}{2005/ 1984, pp.
105--106}) highlights, ``makes it easier for a government to enter into
advantageous agreements,'' while a tarnished reputation imposes costs by
complicating the process of reaching agreements. A favorable reputation
is cultivated not merely by adhering to national interests and values,
but also by demonstrating an understanding of foreign politics and
cultures (\citeproc{ref-cooper2013oxford}{Cooper et al., 2013, p. 2};
\citeproc{ref-mercer2018reputation}{Mercer, 2018};
\citeproc{ref-tetlock1998social}{Tetlock, 1998}). Effective reputation
management therefore requires more than acting on interests and values;
it also involves responding to how external actors believe foreign
audiences interpret their actions. We contend that these insights apply
not only to foreign policy goals but also to the practices used to
pursue them, and not just to states but also to supranational actors
like the EU.

We assume that EU representatives' perceptions of how the EU is viewed
by external counterparts cluster around distinct understandings of the
EU's external role and reputation. These meta-perceptions may emphasize
either the EU's institutional credibility and internal functioning or
its normative ambition and external impact. Existing research on
external perceptions supports this assumption, revealing a wide spectrum
of how the EU is viewed beyond its borders. Much of the literature on
the EU as a global actor has focused on its capacity to act cohesively
and the nature of its actorhood
(\citeproc{ref-bretherton2005}{Bretherton \& Vogler, 2005};
\citeproc{ref-sjostedt1977}{Sjöstedt, 1977}), often stressing the
internal fragmentation between EU institutions and member states
(\citeproc{ref-bretherton2005}{Bretherton \& Vogler, 2005}) and
questioning its effectiveness as a coherent foreign policy actor
(\citeproc{ref-thomas2012still}{Thomas, 2012}). Perceptions have been
shown to differ from country to country, but recurring themes include
the perception of the EU as a leader in trade
(\citeproc{ref-elgstrom_outsiders_2007}{Elgstrom, 2007}) and a secondary
actor in high politics (\citeproc{ref-chaban2014role}{Natalia Chaban \&
Elgström, 2014}). Some portray the EU as a distinctive, normative power
committed to values such as democracy and rule of law
(\citeproc{ref-manners2006}{Manners, 2006}), while others describe it as
a self-interested political player
(\citeproc{ref-hyde2008tragic}{Hyde-Price, 2008}) or even neo-colonial
actor that patronizes its partners
(\citeproc{ref-andretta_imagining_2007}{Andretta \& Doerr, 2007};
\citeproc{ref-bayoumi_egyptian_2007}{Bayoumi, 2007}). This diversity of
external views provides a solid empirical basis for examining variation
in how EU representatives themselves interpret the EU's external image,
and for understanding how those interpretations shape what diplomatic
practices they see as appropriate.

We argue that EU representatives are not only aware of how the EU is
perceived abroad, but actively reflect on how those perceptions should
guide their diplomatic engagement. Like firms or agencies engaged in
public-facing work, they face reputational incentives that influence how
they signal priorities, allocate resources, and choose among available
instruments. Drawing on insights from reputation-sensitive
organizational behavior
(\citeproc{ref-carpenter2012reputation}{Carpenter \& Krause, 2012};
\citeproc{ref-coombs1996communication}{Coombs \& Holladay, 1996}), we
expect EU representatives to act in ways that preserve the Union's
credibility, mitigate reputational risks, and avoid actions that might
further undermine external trust. In practice, this means that
meta-perceptions of the EU's external image can constrain or legitimize
particular forms of engagement. When EU representatives believe the EU
is seen as lacking coherence or credibility, they may retreat from
confrontational tools and prefer practices that are less politically
assertive or reputationally costly. Conversely, when the EU is seen as
normative, effective, and principled, representatives may feel
emboldened to use more visible or assertive instruments. This logic
leads to our first, general hypothesis:

\textbf{H1:} EU officials adjust what diplomatic practices they prefer
according to how they perceive the EU to be perceived by their
counterparts.

Building on our general proposition, we further theorize that EU
representatives' meta-perceptions influence which types of diplomatic
practices they prefer, particularly regarding the degree of
assertiveness or cooperation these practices imply. When EU officials
believe the EU is perceived externally as lacking strategic coherence,
bureaucratically rigid, or normatively overbearing, they may be more
cautious about favoring assertive or punitive tools that could reinforce
such negative images. In this view, reputational sensitivity leads to a
preference for more restrained or dialogical forms of engagement that
are less likely to provoke resistance or reputational backlash. For
instance, if EU representatives believe that their counterparts see the
Union as patronizing or domineering, they may be inclined to avoid
confrontational instruments such as diplomatic sanctions or public
condemnations. Instead, they may emphasize more cooperative practices,
like information exchange, technical assistance, or joint platforms for
dialogue, that promote co-ownership, trust, and mutual understanding.
This reasoning leads to our first sub-hypothesis:

\textbf{H1a:} EU representatives who believe the EU is perceived
externally as lacking credibility or coherence are less likely to prefer
non-cooperative diplomatic instruments.

By contrast, cooperative diplomatic practices tend to be viewed as
constructive, relationship-oriented, and normatively uncontroversial.
Such practices are not only central to the EU's self-image as a
consensus-oriented actor but are also more likely to be perceived as
legitimate and effective across a wide range of reputational contexts.
Even when external perceptions of the EU are critical, cooperative tools
may be seen as appropriate strategies to restore trust, signal openness,
and foster shared goals. Thus, we expect cooperative instruments to
enjoy broad support among EU representatives, regardless of their
interpretation of the EU's external image:

\textbf{H1b:} EU representatives consistently prefer cooperative
diplomatic instruments, regardless of how they believe the EU is
perceived externally.

\section{Empirical Approach}\label{empirical-approach}

We test our hypotheses using original data from an online elite survey
of EU officials and representatives from key institutions (`EU
representatives') who deal broadly with issues related to democracy
support in the Eastern Neighborhood countries (ENCs, i.e.~Armenia,
Azerbaijan, Belarus, Georgia, Moldova, and Ukraine). The region, which
has gained renewed strategic importance in the context of the Russian
invasion of Ukraine, is critical for the EU in several respects:
ensuring energy security (including access to gas and oil and investment
in renewable energy), securing maritime and air routes, and maintaining
stable agricultural trade flows, especially in key commodities such as
grain. Conversely, the EU is important to the ENCs, not only as a major
economic partner but also through longstanding political, social, and
cultural ties. Nonetheless, skepticism about the EU's credibility
persists in parts of the region, particularly among certain state-level
veto players (\citeproc{ref-dimitrova2013}{Dimitrova \& Dragneva,
2013}). Acknowledging such concerns, the
(\citeproc{ref-eucom2011}{Commission, 2011}) emphasized that its revised
approach to neighborhood relations would be ``developed by listening,
not only to the requests for support from partner governments, but also
to demands expressed by civil society.'' This complex relationship makes
the Eastern Neighborhood an ideal setting for examining how EU
representatives' perceptions of how the EU and its democracy promotion
efforts are viewed by external actors shape the ways in which they
engage, formulate policy, and implement democracy support practices.

We conducted the online survey between November 2024 and May 2025,
administered in English, French and German through a newly developed,
open-access, open-source platform that integrates restricted choice
sorting (Q-sorting) . As elites are a hard-to-survey population,
purposive sampling is widely used to efficiently capture relevant
individuals (\citeproc{ref-khoury2020hard}{Khoury, 2020};
\citeproc{ref-walgrave2017surviving}{Walgrave \& Dejaeghere, 2017}). We
identified 1,043 EU representatives through the WhoisWho EU directory,
including staff from the European Commission (N = 375), the European
External Action Service (EEAS, N = 153), the European Parliament (N =
454), the Council of the EU (N = 46), and the European Investment Bank
(EIB, N = 15). To maximize the response rate, we employed multiple
recruitment strategies, including cold-email invitations, follow-up
reminders, phone calls, and direct outreach. For instance, we organized
a webinar with European Commission staff and emphasized the survey's
endorsement by Dr.~Othmar Karas, former Vice-President of the European
Parliament. In total, 61 EU representatives completed the survey,
resulting in a response rate of approximately 5.8 percent. While modest,
this figure aligns with expectations for elite survey research; studies
targeting members of parliament report similarly low or declining
response rates (\citeproc{ref-bailer2014interviews}{Bailer, 2014};
\citeproc{ref-deschouwer2014representing}{Deschouwer \& Depauw, 2014};
\citeproc{ref-hoffmann2008studying}{Hoffmann-Lange, 2008}). To mitigate
social desirability bias and encourage candid participation, all
responses were collected anonymously.

\textbf{Key independent variable: EU external images}

The key independent variable in our analysis is the perception of EU
representatives of the external image of the EU, that is, how they
believe that the EU and its foreign policy are seen by their
counterparts in the European Neighborhood countries. To capture this, we
use a modified version of Q-methodology, a mixed-method approach
originally proposed by Stephenson
(\citeproc{ref-stephenson1935technique}{1935}) and later refined for
social science research by Watts \& Stenner
(\citeproc{ref-watts2012doing}{2012}). In essence, Q-methodology inverts
the logic of conventional (R-method) factor analysis: whereas
traditional factor analysis identifies relationships between variables
across a population of individuals, Q-methodology identifies shared
viewpoints by correlating individuals based on restricted choice ranking
of a common set of statements. These shared views are our ultimate key
independent variable.

Drawing on academic, policy, and media sources, we identified a diverse
set of 42 statements that capture the full range of possible
understandings of how the EU could be perceived from the outside (see
Table 1 in the Appendix). Our primary source for identifying the
statements was the emerging literature on external perceptions of the
EU, which examines how elites and the general public in third countries
view and interpret the Union (\citeproc{ref-chaban2008}{N. Chaban \&
Holland, 2008}; \citeproc{ref-lucarelli_european_2007}{Sonia Lucarelli,
2007}). To complement these academic insights, we incorporate the
self-representation of the EU in official speeches, policy documents,
and foreign policy strategies, and consult experts with specialized
knowledge of EU external relations and the Eastern Neighborhood.
Furthermore, survey respondents were invited to propose any further
statements they believed were missing, allowing us to capture
perspectives beyond those identified through the literature review and
pilot study. We asked the respondents to evaluate the 42 statements by
sorting them on a scale from --5 (strongly disagree) to +5 (strongly
agree), with 0 as a neutral midpoint. Specifically, respondents placed
each statement on a quasinormal distribution grid resembling a bell
curve, based on the assumption that most statements would elicit
moderate agreement or disagreement, while only a few would provoke
strong reactions (\citeproc{ref-brown1980}{Steven Randall Brown, 1980}).
They were invited to watch an animated video explaining the procedure on
how to rate the EU external images in the survey.\footnote{\href{https://www.youtube.com/watch?v=P2NmxqHpt8g}{The
  video is available on YouTube.}}

To minimize acquiescence bias, each statement was formulated in a
positive and a negative version. The respondents were randomly assigned
positive or negative wording for each item, with the overall number of
positive and negative worded statements balanced across the respondents.
To our knowledge, this randomized polarity design, in which respondents
receive randomly assigned positive or negative versions of each item,
represents a novel methodological contribution to the Q-methodology. It
enhances the validity of the data by reducing the influence of framing
effects and ensuring that agreement reflects genuine evaluative
judgments rather than a default tendency to affirm statements
(\citeproc{ref-schuman1996questions}{Schuman \& Presser, 1996}).

This approach is particularly suited to our research goals. First, it
not only reveals what the respondents agree with, but also what they
prioritize, making it particularly effective for capturing subjective
and internally consistent belief systems
(\citeproc{ref-brewer2000}{Brewer et al., 2000};
\citeproc{ref-brown2008}{Steven R. Brown et al., 2008}). Second, because
participants must compare even broadly agreeable statements, the method
discourages the mere repetition of official policy lines and helps
reduce social desirability bias. Third, unlike experimental methods such
as conjoint analysis, our approach elicits stated perceptions in
context, rather than relying on hypothetical scenarios. This provides a
more grounded understanding of how EU representatives interpret the
reputational environment in which they operate. Finally, the technique
enables us to statistically identify clusters of shared perception,
which we use as the main independent variable in testing how different
views of the external image of the EU shape foreign policy practices
(\citeproc{ref-dryzek2008discursive}{Dryzek \& Niemeyer, 2008}).

To identify the shared perception clusters of the EU's external image,
we uncover latent structures in the rankings of the respondents. After
standardizing the responses to ensure comparability, we perform a
hierarchical cluster analysis using the Ward method, which groups the
respondents by minimizing the variance within the cluster at each step.
This approach generates a nested tree of similarity, visualized through
a dendrogram (see Figure 1 in the Appendix), which allows us to detect
clusters of EU representatives with similar viewpoints. Based on the
dendrogram's shape and the pattern of merging distances, we identify a
two-cluster solution that captures the most salient divide in
respondents' interpretations of the EU's external image. A principal
component analysis (PCA) projecting high-dimensional Q-sort data (i.e.,
the 42 statement rankings) onto two principal dimensions supports these
two clusters (see Figure 2 in the Appendix).

Table 1 lists the 15 statements with the largest differences in average
rankings between the two groups of EU representatives in how they
believe the EU is perceived by external actors. Based on the statements
each cluster ranks most positively or negatively (highlighted in bold),
we identify two distinct groups with differing meta-perceptions. The
first group (`Cluster 1') believes the EU is viewed externally as
falling short of its normative ambitions and objectives, but still seen
as a competent institutional actor, capable of balancing internal
dynamics and strategically leveraging its economic power. In contrast,
the second group (`Cluster 2') sees the EU as perceived by its
counterparts more broadly as a credible, effective, and principled
international actor that delivers on its commitments and promotes
stability, democracy, and development.

\usepackage{tabularx}
\begin{table}[htbp]
\centering
\caption{Meta-perceptions of the EU’s external image}
\renewcommand\arraystretch{1.2}
\begin{tabularx}{\textwidth}{l >{\centering\arraybackslash}X >{\centering\arraybackslash}X r r r}
\toprule
Statements (No.) &     Cluster 1 &     Cluster 2  &   Min &  Max & $\Delta$ \\
\midrule
09: EU operates efficiently & -2.64 &  \textbf{1.55} & -2.64 & 1.55 &        4.19 \\
40: EU protects territorial integrity & -1.57 &  \textbf{1.82} & -1.57 & 1.82 &        3.39 \\
36: EU supports civil society credibly &  \textbf{2.29} & -1.05 & -1.05 & 2.29 &        3.33 \\
24: EU provides humanitarian aid & -1.50 &  \textbf{1.68} & -1.50 & 1.68 &        3.18 \\
30: EU offers new market opportunities & -1.43 &  \textbf{1.50} & -1.43 & 1.50 &        2.93 \\
07: EU balances power of member states &  \textbf{3.00} &  0.09 &  0.09 & 3.00 &        2.91 \\
16: EU promotes free trade for poverty/peace &  \textbf{2.50} & -0.36 & -0.36 & 2.50 &        2.86 \\
12: EU promotes democracy/rights credibly & -\textbf{2.43} &  0.32 & -2.43 & 0.32 &        2.75 \\
14: EU uses economic sanctions effectively &  \textbf{2.86} &  0.14 &  0.14 & 2.86 &        2.72 \\
27: EU is politically stabilizing & -1.36 &  \textbf{1.14} & -1.36 & 1.14 &        2.49 \\
04: EU respects cultures  &  0.71 & -1.59 & -1.59 & 0.71 &        2.31 \\
38: EU promotes national sovereignty  &  \textbf{2.36} &  0.18 &  0.18 & 2.36 &        2.18 \\
39: EU mediates conflicts & -1.29 &  0.82 & -1.29 & 0.82 &        2.10 \\
19: EU prioritizes agreements over geopolitics & -\textbf{1.64} &  0.18 & -1.64 & 0.18 &        1.82 \\
37: EU conditionality interferes & -1.36 &  0.45 & -1.36 & 0.45 &        1.81 \\
\bottomrule
\end{tabularx}
\caption*{\small\textit{Note:} Average statement scores by cluster, based on PCA, \textit{N }= 61. Bolded values highlight statements most relevant for interpreting Cluster 1 and Cluster 2, respectively. Statement wording is abbreviated and shown in positive form only. $\Delta$ indicates the absolute difference between clusters.}
\end{table}

\section{Empirical Analysis}\label{empirical-analysis}

\section{Conclusion}\label{conclusion}

When you click the \textbf{Render} button a document will be generated
that includes both content and the output of embedded code. You can
embed code like this:

\begin{Shaded}
\begin{Highlighting}[]
\DecValTok{1} \SpecialCharTok{+} \DecValTok{1}
\end{Highlighting}
\end{Shaded}

\begin{verbatim}
[1] 2
\end{verbatim}

You can add options to executable code like this

\begin{verbatim}
[1] 4
\end{verbatim}

The \texttt{echo:\ false} option disables the printing of code (only
output is displayed).

\phantomsection\label{refs}
\begin{CSLReferences}{1}{0}
\bibitem[\citeproctext]{ref-adler2011}
Adler, E., \& Pouliot, V. (2011). International practices. \emph{Int.
Theory}, \emph{3}(1), 1--36.

\bibitem[\citeproctext]{ref-andretta_imagining_2007}
Andretta, M., \& Doerr, N. (2007). Imagining europe: Internal and
external non-state actors at the european crossroads. \emph{European
Foreign Affairs Review}, \emph{12}(3), 385--400.
\url{https://doi.org/10.54648/EERR2007032}

\bibitem[\citeproctext]{ref-bailer2014interviews}
Bailer, S. (2014). Interviews and surveys in legislative research.
\emph{The Oxford Handbook of Legislative Studies}, 167--193.

\bibitem[\citeproctext]{ref-bayoumi_egyptian_2007}
Bayoumi, S. (2007). Egyptian views of the {EU}: Pragmatic, paternalistic
and partnership concerns. \emph{European Foreign Affairs Review},
\emph{12}(3), 331--347. \url{https://doi.org/10.54648/EERR2007029}

\bibitem[\citeproctext]{ref-bicchi2018}
Bicchi, F., \& Bremberg, N. (2018). European diplomatic practices:
Contemporary challenges and innovative approaches. In \emph{European
diplomacy in practice} (pp. 1--16). Routledge.

\bibitem[\citeproctext]{ref-boulding1959national}
Boulding, K. (1959). National images and international systems.
\emph{Journal of Conflict Resolution}, \emph{3}(2), 120--131.

\bibitem[\citeproctext]{ref-bretherton2005}
Bretherton, C., \& Vogler, J. (2005). \emph{The european union as a
global actor} (2nd ed.). London, England: Routledge.

\bibitem[\citeproctext]{ref-brewer2000}
Brewer, G. A., Selden, S. C., \& Facer, R. L., II. (2000). Individual
conceptions of public service motivation. \emph{Public Adm. Rev.},
\emph{60}(3), 254--264.

\bibitem[\citeproctext]{ref-brown1980}
Brown, Steven Randall. (1980). Political subjectivity: Applications of q
methodology in political science. \emph{(No Title)}.

\bibitem[\citeproctext]{ref-brown2008}
Brown, Steven R., Durning, D. W., \& Selden, S. C. (2008). Q
methodology. \emph{PUBLIC ADMINISTRATION AND PUBLIC POLICY-NEW YORK-},
\emph{134}, 721.

\bibitem[\citeproctext]{ref-carpenter2012reputation}
Carpenter, D. P., \& Krause, G. A. (2012). Reputation and public
administration. \emph{Public Administration Review}, \emph{72}(1),
26--32.

\bibitem[\citeproctext]{ref-chaban2014role}
Chaban, Natalia, \& Elgström, O. (2014). The role of the EU in an
emerging new world order in the eyes of the chinese, indian and russian
press. \emph{Journal of European Integration}, \emph{36}(2), 170--188.

\bibitem[\citeproctext]{ref-chaban2008}
Chaban, N., \& Holland, M. (2008). \emph{The european union and the
asia-pacific: Media, public and elite perceptions of the EU}. Taylor \&
Francis. Retrieved from
\url{https://books.google.ch/books?id=Bbhwlwxxc0YC}

\bibitem[\citeproctext]{ref-chaban_east_2007}
Chaban, Natalia, \& Kauffmann, M. (2007). 'East is east, and west is
west': A survey of {EU} images in japan's public discourses.
\emph{European Foreign Affairs Review}, \emph{12}(3), 363--384.
\url{https://doi.org/10.54648/EERR2007031}

\bibitem[\citeproctext]{ref-eucom2011}
Commission, E. (2011). \emph{JOINT COMMUNICATION TO THE EUROPEAN
COUNCIL, THE EUROPEAN PARLIAMENT, THE COUNCIL, THE EUROPEAN ECONOMIC AND
SOCIAL COMMITTEE AND THE COMMITTEE OF THE REGIONS a PARTNERSHIP FOR
DEMOCRACY AND SHARED PROSPERITY WITH THE SOUTHERN MEDITERRANEAN,
COM:2011:0200:FIN:en:PDF}. European Commission.

\bibitem[\citeproctext]{ref-coombs1996communication}
Coombs, W. T., \& Holladay, S. J. (1996). Communication and attributions
in a crisis: An experimental study in crisis communication.
\emph{Journal of Public Relations Research}, \emph{8}(4), 279--295.

\bibitem[\citeproctext]{ref-cooper2013oxford}
Cooper, A. F., Heine, J., \& Thakur, R. (2013). \emph{The oxford
handbook of modern diplomacy}. OUP Oxford.

\bibitem[\citeproctext]{ref-dafoe2014reputation}
Dafoe, A., Renshon, J., \& Huth, P. (2014). Reputation and status as
motives for war. \emph{Annual Review of Political Science}, \emph{17},
371--393.

\bibitem[\citeproctext]{ref-delcour2013meandering}
Delcour, L. (2013). Meandering europeanisation. EU policy instruments
and policy convergence in georgia under the eastern partnership.
\emph{East European Politics}, \emph{29}(3), 344--357.

\bibitem[\citeproctext]{ref-delreux2021ego}
Delreux, T., \& Pipart, F. (2021). Ego versus alter: Internal and
external perceptions of the EU's role in global environmental
negotiations. \emph{Journal of Common Market Studies}, \emph{59}(5),
1284--1302.

\bibitem[\citeproctext]{ref-deschouwer2014representing}
Deschouwer, K., \& Depauw, S. (2014). \emph{Representing the people: A
survey among members of statewide and substate parliaments}. Oxford
University Press.

\bibitem[\citeproctext]{ref-dimitrova2013}
Dimitrova, A., \& Dragneva, R. (2013). Shaping convergence with the {EU}
in foreign policy and state aid in post-orange ukraine: Weak external
incentives, powerful veto players. \emph{Eur. Asia. Stud.},
\emph{65}(4), 658--681.

\bibitem[\citeproctext]{ref-dryzek2008discursive}
Dryzek, J., \& Niemeyer, S. (2008). Discursive representation.
\emph{American Political Science Review}, \emph{102}(4), 481--493.

\bibitem[\citeproctext]{ref-elgstrom_outsiders_2007}
Elgstrom, O. (2007). Outsiders' perceptions of the european union in
international trade negotiations. \emph{Journal of Common Market
Studies}, \emph{45}(4), 949--967.

\bibitem[\citeproctext]{ref-hermann1967attempt}
Hermann, C., \& Hermann, M. (1967). An attempt to simulate the outbreak
of world war i. \emph{American Political Science Review}, \emph{61}(2),
400--416.

\bibitem[\citeproctext]{ref-hoffmann2008studying}
Hoffmann-Lange, U. (2008). \emph{Studying elite vs mass opinion}. Sage.

\bibitem[\citeproctext]{ref-holsti1970national}
Holsti, K. (1970). National role conceptions in the study of foreign
policy. \emph{International Studies Quarterly}, \emph{14}(3), 233--309.

\bibitem[\citeproctext]{ref-holsti1962belief}
Holsti, O. (1962). The belief system and national images: A case study.
\emph{Journal of Conflict Resolution}, \emph{6}(3), 244--252.

\bibitem[\citeproctext]{ref-hyde2008tragic}
Hyde-Price, A. (2008). A'tragic actor'? A realist perspective on'ethical
power europe'. \emph{International Affairs}, 29--44.

\bibitem[\citeproctext]{ref-jervis1976}
Jervis, R. (1976). \emph{Misperception in international politics}.
Princeton University Press.

\bibitem[\citeproctext]{ref-jervis1989logic}
Jervis, R. (1989). \emph{The logic of images in international
relations}. Columbia University Press.

\bibitem[\citeproctext]{ref-keohane2005after}
Keohane, R. (2005/ 1984). \emph{After hegemony: Cooperation and discord
in the world political economy}. Princeton University Press.

\bibitem[\citeproctext]{ref-khoury2020hard}
Khoury, R. (2020). Hard-to-survey populations and respondent-driven
sampling: Expanding the political science toolbox. \emph{Perspectives on
Politics}, \emph{18}(2), 509--526.

\bibitem[\citeproctext]{ref-lucarelli_european_2007}
Lucarelli, Sonia. (2007). The european union in the eyes of others:
Towards filling a gap in the literature. \emph{European Foreign Affairs
Review}, \emph{12}(3), 249--270.

\bibitem[\citeproctext]{ref-lucarelli_seen_2014}
Lucarelli, Sonia. (2014). Seen from the outside: The state of the art on
the external image of the {EU}. \emph{Journal of European Integration},
\emph{36}(1), 1--16.

\bibitem[\citeproctext]{ref-lucarelli2010}
Lucarelli, S., \& Fioramonti, L. (2010). \emph{External perceptions of
the european union as a global actor}. Taylor \& Francis. Retrieved from
\url{https://books.google.ch/books?id=zkWOAgAAQBAJ}

\bibitem[\citeproctext]{ref-manners2002normative}
Manners, I. (2002). Normative power europe: A contradiction in terms?
\emph{Journal of Common Market Studies}, \emph{40}(2), 235--258.

\bibitem[\citeproctext]{ref-manners2006}
Manners, I. (2006). European union, normative power and ethical foreign
policy. \emph{Rethinking Ethical Foreign Policy: Pitfalls, Possibilities
and Paradoxes}, \emph{116}.

\bibitem[\citeproctext]{ref-mercer2018reputation}
Mercer, J. (2018). \emph{Reputation and international politics}. Cornell
University Press.

\bibitem[\citeproctext]{ref-nye2004soft}
Nye, J. (2004). \emph{Soft power: The means to success in world
politics}. Public Affairs.

\bibitem[\citeproctext]{ref-pace2014eu}
Pace, M. (2014). The EU's interpretation of the {``arab uprisings''}:
Understanding the different visions about democratic change in EU-MENA
relations. \emph{Journal of Common Market Studies}, \emph{52}(5),
969--984.

\bibitem[\citeproctext]{ref-pouliot2010}
Pouliot, V. (2010). \emph{International security in practice: The
politics of NATO-russia diplomacy} (Vol. 113). Cambridge University
Press.

\bibitem[\citeproctext]{ref-renshon2008theory}
Renshon, J., \& Renshon, S. A. (2008). The theory and practice of
foreign policy decision making. \emph{Political Psychology},
\emph{29}(4), 509--536.

\bibitem[\citeproctext]{ref-scheipers_normative_2007}
Scheipers, S., \& Sicurelli, D. (2007). Normative power europe: A
credible utopia? \emph{Journal of Common Market Studies}, \emph{45}(2),
435--457. \url{https://doi.org/10.1111/j.1468-5965.2007.00717.x}

\bibitem[\citeproctext]{ref-schlipphak_action_2013}
Schlipphak, B. (2013). Action and attitudes matter: International public
opinion towards the european union. \emph{European Union Politics},
\emph{14}(4), 590--618. \url{https://doi.org/10.1177/1465116513482527}

\bibitem[\citeproctext]{ref-schuman1996questions}
Schuman, H., \& Presser, S. (1996). \emph{Questions and answers in
attitude surveys: Experiments on question form, wording, and context}.
Sage.

\bibitem[\citeproctext]{ref-shapiro1973cognitive}
Shapiro, M., \& Bonham, M. (1973). Cognitive process and foreign policy
decision-making. \emph{International Studies Quarterly}, \emph{17}(2),
147--174.

\bibitem[\citeproctext]{ref-sjostedt1977}
Sjöstedt, G. (1977). The exercise of international civil power.
\emph{Coop. Confl.}, \emph{12}(1), 21--39.

\bibitem[\citeproctext]{ref-smith2011enlargement}
Smith, K. E. (2011). Enlargement, the neighbourhood, and european order.
\emph{International Relations and the European Union}, 299--323.

\bibitem[\citeproctext]{ref-smith2014european}
Smith, K. E. (2014). Is the european union's soft power in decline?
\emph{Current History}, \emph{113}(761), 104.

\bibitem[\citeproctext]{ref-sprout1957environmental}
Sprout, H., \& Sprout, M. (1957). Environmental factors in the study of
international politics. \emph{Conflict Resolution}, \emph{1}(4),
309--328.

\bibitem[\citeproctext]{ref-stephenson1935technique}
Stephenson, W. (1935). Technique of factor analysis. \emph{Nature},
\emph{136}(3434), 297--297.

\bibitem[\citeproctext]{ref-stephenson1953}
Stephenson, W. (1953). \emph{The study of behavior; q-technique and its
methodology.}

\bibitem[\citeproctext]{ref-tetlock1998social}
Tetlock, P. (1998). \emph{Social psychology and world politics.}
McGraw-Hill.

\bibitem[\citeproctext]{ref-thomas2012still}
Thomas, D. C. (2012). Still punching below its weight? Coherence and
effectiveness in european union foreign policy. \emph{JCMS: Journal of
Common Market Studies}, \emph{50}(3), 457--474.

\bibitem[\citeproctext]{ref-torney2014external}
Torney, D. (2014). External perceptions and EU foreign policy
effectiveness: The case of climate change. \emph{Journal of Common
Market Studies}, \emph{52}(6), 1358--1373.

\bibitem[\citeproctext]{ref-walgrave2017surviving}
Walgrave, S., \& Dejaeghere, Y. (2017). Surviving information overload:
How elite politicians select information. \emph{Governance},
\emph{30}(2), 229--244.

\bibitem[\citeproctext]{ref-watts2012doing}
Watts, S., \& Stenner, P. (2012). \emph{Doing q methodological research:
Theory, method \& interpretation}.

\bibitem[\citeproctext]{ref-wish1980foreign}
Wish, N. B. (1980). Foreign policy makers and their national role
conceptions. \emph{International Studies Quarterly}, \emph{24}(4),
532--554.

\end{CSLReferences}




\end{document}
